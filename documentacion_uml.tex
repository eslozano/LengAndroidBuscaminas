% !TEX TS-program = pdflatex
% !TEX encoding = UTF-8 Unicode

% This is a simple template for a LaTeX document using the "article" class.
% See "book", "report", "letter" for other types of document.

\documentclass[11pt]{article} % use larger type; default would be 10pt

\usepackage[utf8]{inputenc} % set input encoding (not needed with XeLaTeX)

%%% Examples of Article customizations
% These packages are optional, depending whether you want the features they provide.
% See the LaTeX Companion or other references for full information.

%%% PAGE DIMENSIONS
\usepackage{geometry} % to change the page dimensions
\geometry{a4paper} % or letterpaper (US) or a5paper or....
% \geometry{margin=2in} % for example, change the margins to 2 inches all round
% \geometry{landscape} % set up the page for landscape
%   read geometry.pdf for detailed page layout information

\usepackage{graphicx} % support the \includegraphics command and options

% \usepackage[parfill]{parskip} % Activate to begin paragraphs with an empty line rather than an indent

%%% PACKAGES
\usepackage{booktabs} % for much better looking tables
\usepackage{array} % for better arrays (eg matrices) in maths
\usepackage{paralist} % very flexible & customisable lists (eg. enumerate/itemize, etc.)
\usepackage{verbatim} % adds environment for commenting out blocks of text & for better verbatim
\usepackage{subfig} % make it possible to include more than one captioned figure/table in a single float
% These packages are all incorporated in the memoir class to one degree or another...

%%% HEADERS & FOOTERS
\usepackage{fancyhdr} % This should be set AFTER setting up the page geometry
\pagestyle{fancy} % options: empty , plain , fancy
\renewcommand{\headrulewidth}{0pt} % customise the layout...
\lhead{}\chead{}\rhead{}
\lfoot{}\cfoot{\thepage}\rfoot{}

%%% SECTION TITLE APPEARANCE
\usepackage{sectsty}
\allsectionsfont{\sffamily\mdseries\upshape} % (See the fntguide.pdf for font help)
% (This matches ConTeXt defaults)

%%% ToC (table of contents) APPEARANCE
\usepackage[nottoc,notlof,notlot]{tocbibind} % Put the bibliography in the ToC
\usepackage[titles,subfigure]{tocloft} % Alter the style of the Table of Contents
\renewcommand{\cftsecfont}{\rmfamily\mdseries\upshape}
\renewcommand{\cftsecpagefont}{\rmfamily\mdseries\upshape} % No bold!

%%% END Article customizations

%%% The "real" document content comes below...

\title{Documentación UML }
\author{Lozano, Alvarado, Lasso}
%\date{} % Activate to display a given date or no date (if empty),
         % otherwise the current date is printed 

\begin{document}
\maketitle

\section{Casos de Uso}
Caso de Uso: Iniciar el juego. 
\\Descripción: 
En este caso de uso describimos los posibles escenarios que se puedan presentar cuando el usuario inicia el juego. El usuario puede escoger entre jugar, revisar puntajes altos, ver la ayuda y salir del juego. \
\\  \

Caso de Uso: Seleccionar nivel de dificultad del juego. 
\\Descripción: 
En este caso de uso describimos los posibles escenarios que se puedan presentar cuando el usuario selecciona un nivel de dificultad o personalizar el tamaño del tablero.
Según el nivel de dificultad que el usuario escoja se presentará un tablero con un tamaño y un número de minas específico.\
\\  \

Caso de Uso: Descubrir celda. 
\\Descripción: 
En este caso de uso describimos los posibles escenarios que se puedan presentar cuando el usuario descubre una celda, esto lo hace por medio de un tap en la pantalla. \
\\  \ 
 
Caso de Uso: Marcar celda con una bandera . 
\\Descripción: 
En este caso de uso describimos los posibles escenarios que se puedan presentar cuando el usuario selecciona una bandera y la arrastra a una celda determinada, tomamos en cuenta como escenario exitoso que la celda a la que vamos a arrastrar la bandera contiene una mina.\
\\ \

\section{Escenarios}

Caso de uso 1: Iniciar el juego
\\Escenario 1.1: El usuario selecciona jugar.\
\\Actores:Usuario.\
\\Asunciones: El usuario pudo ingresar correctamente al juego.\
\\Acciones:\
\begin{itemize}
\item El juego muestra cuatro opciones Juga, Revisar puntajes altos, Ver la ayuda o Salir.
\item El usuario seleccionó la opción de Jugar.
\item El juego muestra un submenu en el que el usuario debera escoger la dificultad deseada.
\end {itemize}
Resultados:
El juego muestra un submenu de para escoger la dificultad
\\   \
Caso de uso 1: Iniciar el juego
\\Escenario 1.2: El usuario selecciona Revisar puntajes altos.\
\\Actores:Usuario.\
\\Asunciones: El usuario pudo ingresar correctamente al juego.\
\\Acciones:\
\begin{itemize}
\item El juego muestra cuatro opciones Juga, Revisar puntajes altos, Ver la ayuda o Salir.
\item El usuario seleccionó la opción de Revisar puntajes altos.
\item El juego muestra un listado de los mejores tiempos por dificultad.
\end {itemize}
Resultados:
El juego muestra un listado de los mejores puntajes por dificultad
\\   \
Caso de uso 1: Iniciar el juego
\\Escenario 1.3: El usuario selecciona Ver la ayuda.\
\\Actores:Usuario.\
\\Asunciones: El usuario pudo ingresar correctamente al juego.\
\\Acciones:\
\begin{itemize}
\item El juego muestra cuatro opciones Juga, Revisar puntajes altos, Ver la ayuda o Salir.
\item El usuario seleccionó la opción de Ver la ayuda.
\item El juego muestra un grafico indicando las reglas basicas del juego.
\end {itemize}
Resultados:
El juego muestra un grafico indicando las reglas basicas del juego
\\   \

Caso de uso 2: Seleccionar nivel de dificultad del juego
\\Escenario 2.1: El usuario selecciona el nivel de dificultad principiante.\
\\Actores:Usuario.\
\\Asunciones: El usuario pudo ingresar correctamente al juego.\
\\Acciones:\
\begin{itemize}
\item El juego muestra dos opciones Jugar o Salir.
\item El usuario seleccionó la opción de jugar.
\item El juego muestra un submenu en el que el usuario debera escoger Principiante, Intermedio o Avanzado.
\item El usuario seleccionó el nivel de dificultad principiante.
\end {itemize}
Resultados:
El juego muestra un tablero de dimensión 9x9 con 10 minas distribuidas aleatoriamente
\\   \
\\Caso de uso 2: Seleccionar nivel de dificultad del juego\
\\Escenario 2.2: El usuario selecciona el nivel de dificultad Intermedio.\
\\Actores:Usuario.\
\\Asunciones: El usuario pudo ingresar correctamente al juego.\
\\Acciones:\
\begin{itemize}
\item El juego muestra dos opciones Jugar o Salir.
\item El usuario seleccionó la opción de jugar.
\item El juego muestra un submenu en el que el usuario deberá escoger Principiante, Intermedio o Avanzado.
\item El usuario seleccionó el nivel de dificultad Intermedio.
\end {itemize}
Resultados:
El juego muestra un tablero de dimensión 16x16 con 40 minas distribuidas aleatoriamente.
\\ \
\\Caso de uso 2: Seleccionar nivel de dificultad del juego\
\\Escenario 2.3: El usuario selecciona el nivel de dificultad Avanzado.\
\\Actores:Usuario.\
\\Asunciones: El usuario pudo ingresar correctamente al juego.\
\\Acciones:\
\begin{itemize}
\item El juego muestra dos opciones Jugar o Salir.
\item El usuario seleccionó la opción de jugar.
\item El juego muestra un submenu en el que el usuario deberá escoger Principiante, Intermedio o Avanzado.
\item El usuario seleccionó el nivel de dificultad Avanzado.
\end {itemize}
Resultados:
El juego muestra un tablero de dimensión 16x30 con 99 minas distribuidas aleatoriamente.
\\ \

Caso de uso 3: Marcar celda con una bandera
\\Escenario 3.1: La celda es marcada  con una bandera de manera exitosa .\
\\Actores:Usuario.\
\\Asunciones: 
\begin{itemize}
\item El usuario pudo ingresar correctamente al juego.
\item La ceda donde vamos a colocar la bandera no ha sido descubierta
\item Contenido de la celda donde vamos a colocar la bandera es una mina.
\end {itemize}
Acciones:
\begin{itemize}
\item El juego muestra junto al tablero un icono donde se encuentra dibujada una bandera.
\item El usuario selecciona el icono con la bandera y la arrastra a una celda del tablero. 
\end {itemize}
Resultados:
La celda escogida por el usuario fue marcada con una bandera.
\\ \

Caso de uso 4: Descubrir celda\
\\Escenario 4.1: El usuario descubre una celda con numero.\
\\Actores:Usuario.\
\\Asunciones: 
\begin{itemize}
\item El usuario pudo ingresar correctamente al juego.
\item La ceda que se va a descubrir no ha sido descubierta
\end {itemize}
Acciones:
\begin{itemize}
\item El juego muestra un tablero con celdas.
\item El usuario da tap a una celda con número. 
\end {itemize}
Resultados:
La celda escogida por el usuario es descubierta
\\ \

Escenario 4.2: El usuario descubre una celda en blanco.\
Actores:Usuario.\
Asunciones: 
\begin{itemize}
\item El usuario pudo ingresar correctamente al juego.
\item La ceda que se va a descubrir no ha sido descubierta
\end {itemize}
Acciones:
\begin{itemize}
\item El juego muestra un tablero con celdas.
\item El usuario da tap a una celda en blanco. 
\end {itemize}
Resultados:
La celda escogida por el usuario es descubierta y ademas las celdas alrededor tambien son descubiertas
\\ \

Escenario 4.3: El usuario descubre una celda con mina.\
Actores:Usuario.\
Asunciones: 
\begin{itemize}
\item El usuario pudo ingresar correctamente al juego.
\item La ceda que se va a descubrir no ha sido descubierta
\end {itemize}
Acciones:
\begin{itemize}
\item El juego muestra un tablero con celdas.
\item El usuario da tap a una celda con mina. 
\end {itemize}
Resultados:
Se muestran todas las minas del tablero
El usuario pierde el juego
\\ \




\end{document}
